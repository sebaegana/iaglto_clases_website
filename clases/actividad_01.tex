% Options for packages loaded elsewhere
% Options for packages loaded elsewhere
\PassOptionsToPackage{unicode}{hyperref}
\PassOptionsToPackage{hyphens}{url}
\PassOptionsToPackage{dvipsnames,svgnames,x11names}{xcolor}
%
\documentclass[
  letterpaper,
  DIV=11,
  numbers=noendperiod]{scrartcl}
\usepackage{xcolor}
\usepackage{amsmath,amssymb}
\setcounter{secnumdepth}{-\maxdimen} % remove section numbering
\usepackage{iftex}
\ifPDFTeX
  \usepackage[T1]{fontenc}
  \usepackage[utf8]{inputenc}
  \usepackage{textcomp} % provide euro and other symbols
\else % if luatex or xetex
  \usepackage{unicode-math} % this also loads fontspec
  \defaultfontfeatures{Scale=MatchLowercase}
  \defaultfontfeatures[\rmfamily]{Ligatures=TeX,Scale=1}
\fi
\usepackage{lmodern}
\ifPDFTeX\else
  % xetex/luatex font selection
\fi
% Use upquote if available, for straight quotes in verbatim environments
\IfFileExists{upquote.sty}{\usepackage{upquote}}{}
\IfFileExists{microtype.sty}{% use microtype if available
  \usepackage[]{microtype}
  \UseMicrotypeSet[protrusion]{basicmath} % disable protrusion for tt fonts
}{}
\makeatletter
\@ifundefined{KOMAClassName}{% if non-KOMA class
  \IfFileExists{parskip.sty}{%
    \usepackage{parskip}
  }{% else
    \setlength{\parindent}{0pt}
    \setlength{\parskip}{6pt plus 2pt minus 1pt}}
}{% if KOMA class
  \KOMAoptions{parskip=half}}
\makeatother
% Make \paragraph and \subparagraph free-standing
\makeatletter
\ifx\paragraph\undefined\else
  \let\oldparagraph\paragraph
  \renewcommand{\paragraph}{
    \@ifstar
      \xxxParagraphStar
      \xxxParagraphNoStar
  }
  \newcommand{\xxxParagraphStar}[1]{\oldparagraph*{#1}\mbox{}}
  \newcommand{\xxxParagraphNoStar}[1]{\oldparagraph{#1}\mbox{}}
\fi
\ifx\subparagraph\undefined\else
  \let\oldsubparagraph\subparagraph
  \renewcommand{\subparagraph}{
    \@ifstar
      \xxxSubParagraphStar
      \xxxSubParagraphNoStar
  }
  \newcommand{\xxxSubParagraphStar}[1]{\oldsubparagraph*{#1}\mbox{}}
  \newcommand{\xxxSubParagraphNoStar}[1]{\oldsubparagraph{#1}\mbox{}}
\fi
\makeatother


\usepackage{longtable,booktabs,array}
\usepackage{calc} % for calculating minipage widths
% Correct order of tables after \paragraph or \subparagraph
\usepackage{etoolbox}
\makeatletter
\patchcmd\longtable{\par}{\if@noskipsec\mbox{}\fi\par}{}{}
\makeatother
% Allow footnotes in longtable head/foot
\IfFileExists{footnotehyper.sty}{\usepackage{footnotehyper}}{\usepackage{footnote}}
\makesavenoteenv{longtable}
\usepackage{graphicx}
\makeatletter
\newsavebox\pandoc@box
\newcommand*\pandocbounded[1]{% scales image to fit in text height/width
  \sbox\pandoc@box{#1}%
  \Gscale@div\@tempa{\textheight}{\dimexpr\ht\pandoc@box+\dp\pandoc@box\relax}%
  \Gscale@div\@tempb{\linewidth}{\wd\pandoc@box}%
  \ifdim\@tempb\p@<\@tempa\p@\let\@tempa\@tempb\fi% select the smaller of both
  \ifdim\@tempa\p@<\p@\scalebox{\@tempa}{\usebox\pandoc@box}%
  \else\usebox{\pandoc@box}%
  \fi%
}
% Set default figure placement to htbp
\def\fps@figure{htbp}
\makeatother





\setlength{\emergencystretch}{3em} % prevent overfull lines

\providecommand{\tightlist}{%
  \setlength{\itemsep}{0pt}\setlength{\parskip}{0pt}}



 


\KOMAoption{captions}{tableheading}
\makeatletter
\@ifpackageloaded{caption}{}{\usepackage{caption}}
\AtBeginDocument{%
\ifdefined\contentsname
  \renewcommand*\contentsname{Table of contents}
\else
  \newcommand\contentsname{Table of contents}
\fi
\ifdefined\listfigurename
  \renewcommand*\listfigurename{List of Figures}
\else
  \newcommand\listfigurename{List of Figures}
\fi
\ifdefined\listtablename
  \renewcommand*\listtablename{List of Tables}
\else
  \newcommand\listtablename{List of Tables}
\fi
\ifdefined\figurename
  \renewcommand*\figurename{Figure}
\else
  \newcommand\figurename{Figure}
\fi
\ifdefined\tablename
  \renewcommand*\tablename{Table}
\else
  \newcommand\tablename{Table}
\fi
}
\@ifpackageloaded{float}{}{\usepackage{float}}
\floatstyle{ruled}
\@ifundefined{c@chapter}{\newfloat{codelisting}{h}{lop}}{\newfloat{codelisting}{h}{lop}[chapter]}
\floatname{codelisting}{Listing}
\newcommand*\listoflistings{\listof{codelisting}{List of Listings}}
\makeatother
\makeatletter
\makeatother
\makeatletter
\@ifpackageloaded{caption}{}{\usepackage{caption}}
\@ifpackageloaded{subcaption}{}{\usepackage{subcaption}}
\makeatother
\makeatletter
\@ifpackageloaded{tcolorbox}{}{\usepackage[skins,breakable]{tcolorbox}}
\makeatother
\makeatletter
\@ifundefined{shadecolor}{\definecolor{shadecolor}{HTML}{FF0000}}{}
\makeatother
\makeatletter
\makeatother
\makeatletter
\ifdefined\Shaded\renewenvironment{Shaded}{\begin{tcolorbox}[frame hidden, sharp corners, interior hidden, borderline west={3pt}{0pt}{shadecolor}, boxrule=0pt, enhanced, breakable]}{\end{tcolorbox}}\fi
\makeatother
\makeatletter
\@ifpackageloaded{fontawesome5}{}{\usepackage{fontawesome5}}
\makeatother
\usepackage{bookmark}
\IfFileExists{xurl.sty}{\usepackage{xurl}}{} % add URL line breaks if available
\urlstyle{same}
\hypersetup{
  pdfauthor={Sebastián Egaña Santibáñez },
  colorlinks=true,
  linkcolor={blue},
  filecolor={Maroon},
  citecolor={Blue},
  urlcolor={Blue},
  pdfcreator={LaTeX via pandoc}}


\title{\includegraphics[width=5cm]{Imagen5.png}\\
ECO5008 Modelos predictivos}
\usepackage{etoolbox}
\makeatletter
\providecommand{\subtitle}[1]{% add subtitle to \maketitle
  \apptocmd{\@title}{\par {\large #1 \par}}{}{}
}
\makeatother
\subtitle{Actividad: Enfoque de Decision Science}
\author{Sebastián Egaña Santibáñez
\href{mailto:segana@fen.uchile.cl}{\faIcon{inbox}}}
\date{}
\begin{document}
\maketitle


\begin{center}\rule{0.5\linewidth}{0.5pt}\end{center}

\section{Enlaces del profesor}\label{enlaces-del-profesor}

\href{https://segana.netlify.app}{\faIcon{link}}
https://segana.netlify.app

\href{https://github.com/sebaegana}{\faIcon{github}}
https://github.com/sebaegana

\href{https://www.linkedin.com/in/sebastian-egana-santibanez/}{\faIcon{linkedin}}
https://www.linkedin.com/in/sebastian-egana-santibanez/

\begin{center}\rule{0.5\linewidth}{0.5pt}\end{center}

\section{Aplicando el enfoque de Decision
Science}\label{aplicando-el-enfoque-de-decision-science}

\subsection{Objetivo de la actividad}\label{objetivo-de-la-actividad}

Aplicar el marco de \textbf{Decision Science} a \textbf{uno de los
análisis vistos en el curso} (por ejemplo):

\begin{itemize}
\tightlist
\item
  Pronóstico de demanda por causas respiratoria
\item
  Weibull de cirugías cardíacas
\item
  Weibull de maquinas de hemodiálisis
\item
  EOQ e inventario con incertidumbre
\end{itemize}

\subsection{Frame de Decision Science}\label{frame-de-decision-science}

\begin{enumerate}
\def\labelenumi{\arabic{enumi}.}
\tightlist
\item
  \textbf{Definir el problema}
\item
  \textbf{Reunir y analizar datos}
\item
  \textbf{Desarrollar y evaluar alternativas}
\item
  \textbf{Seleccionar e implementar soluciones}
\end{enumerate}

Veamos que se debe entregar por aca apartado:

\subsection{1. Definir el problema}\label{definir-el-problema}

\begin{quote}
Explica \textbf{qué decisión} busca apoyar tu modelo y \textbf{por qué
es importante}.
\end{quote}

\textbf{Ejemplos de preguntas guía:}

\begin{itemize}
\tightlist
\item
  ¿Qué decisión de negocio, clínica o logística se desea mejorar?
\item
  ¿Cuál es el objetivo medible? (ej: reducir costos, predecir egresos,
  optimizar turnos)
\item
  ¿Qué riesgos o restricciones deben considerarse?
\end{itemize}

\textbf{Entrega esperada:}

Un párrafo con la definición del problema y los objetivos de la
decisión. Se debe explicar si existe algún modelo relacionado y las
posibles alternativas para evaluar.

\subsection{2. Reunir y analizar datos}\label{reunir-y-analizar-datos}

\begin{quote}
Describe \textbf{qué información} respalda la decisión y \textbf{cómo la
analizaste}.
\end{quote}

\textbf{Incluye:} - Datos entregados por el material del curso - Otras
posibles fuentes a utilizar

\textbf{Entrega esperada:}

Un resumen con los datos, metodología y resultados clave.

\subsection{3. Desarrollar y evaluar
alternativa}\label{desarrollar-y-evaluar-alternativa}

\begin{quote}
Muestra \textbf{cómo los resultados del modelo generan opciones de
decisión.}
\end{quote}

\textbf{Ejemplos:}

\begin{itemize}
\tightlist
\item
  Escenarios vistos en clases
\item
  Posibilidad de implementar algún otro tipo de escenario (utilizar
  algún tipo de IA)
\end{itemize}

\textbf{Entrega esperada:}

Tabla o gráfico comparativo de alternativas y criterios de evaluación.

\subsection{4. Seleccionar e implementar
soluciones}\label{seleccionar-e-implementar-soluciones}

\begin{quote}
Explica \textbf{qué alternativa elegirías} y \textbf{cómo la pondrías en
práctica.}
\end{quote}

\textbf{Preguntas guía:}

\begin{itemize}
\tightlist
\item
  ¿Qué alternativa ofrece mejor equilibrio entre costo y beneficio?
\item
  ¿Cómo se implementaría en la organización o sistema?
\item
  ¿Qué indicadores se usarán para monitorear su éxito?
\item
  ¿Qué revisión o retroalimentación se recomienda?
\end{itemize}

\textbf{Entrega esperada:}

Propuesta final y breve plan de implementación.

\subsection{Entregables del grupo}\label{entregables-del-grupo}

Breve resumen (máx. 1 página) con:

\begin{itemize}
\tightlist
\item
  Definición de problema
\item
  Reunir y analizar datos
\item
  Desarrollar y evaluar alternativas
\item
  Seleccionar e implementar soluciones
\item
  Comentarios y recomendación final
\end{itemize}

\subsection{Criterios de evaluación}\label{criterios-de-evaluaciuxf3n}

\begin{longtable}[]{@{}
  >{\raggedright\arraybackslash}p{(\linewidth - 6\tabcolsep) * \real{0.1964}}
  >{\raggedright\arraybackslash}p{(\linewidth - 6\tabcolsep) * \real{0.2857}}
  >{\raggedright\arraybackslash}p{(\linewidth - 6\tabcolsep) * \real{0.2679}}
  >{\raggedright\arraybackslash}p{(\linewidth - 6\tabcolsep) * \real{0.2500}}@{}}
\toprule\noalign{}
\begin{minipage}[b]{\linewidth}\raggedright
Criterio
\end{minipage} & \begin{minipage}[b]{\linewidth}\raggedright
Excelente (3)
\end{minipage} & \begin{minipage}[b]{\linewidth}\raggedright
Aceptable (2)
\end{minipage} & \begin{minipage}[b]{\linewidth}\raggedright
Parcial (1)
\end{minipage} \\
\midrule\noalign{}
\endhead
\bottomrule\noalign{}
\endlastfoot
Problema y objetivo & Claros y medibles & Parcialmente definidos &
Vaguedad o sin métrica \\
Datos y análisis & Bien sustentados, relevantes & Adecuados pero
limitados & Escasos o poco pertinentes \\
Alternativas y evaluación & Múltiples, con evidencia comparativa & Una
sola alternativa o evaluación superficial & Sin análisis cuantitativo \\
Recomendación final & Clara, implementable y sustentada & Parcialmente
implementable & Ambigua o sin acción concreta \\
\end{longtable}

\subsection{Cierre}\label{cierre}

\begin{quote}
``El valor de un modelo no está solo en predecir, sino en \textbf{ayudar
a decidir mejor}.''
\end{quote}

Cada equipo debe demostrar cómo sus resultados \textbf{informan una
decisión real},\\
siguiendo la lógica de \textbf{Decision Science: datos → análisis →
decisión → acción.}




\end{document}
